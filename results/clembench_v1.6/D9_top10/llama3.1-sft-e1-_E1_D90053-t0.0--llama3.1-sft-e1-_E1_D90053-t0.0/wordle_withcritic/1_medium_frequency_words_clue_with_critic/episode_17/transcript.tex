
\documentclass{article}
\usepackage{colortbl}
\usepackage{makecell}
\usepackage{multirow}
\usepackage{supertabular}

\begin{document}

\newcounter{utterance}

\twocolumn

{ \footnotesize  \setcounter{utterance}{1}
\setlength{\tabcolsep}{0pt}
\begin{supertabular}{c@{$\;$}|p{.15\linewidth}@{}p{.15\linewidth}p{.15\linewidth}p{.15\linewidth}p{.15\linewidth}p{.15\linewidth}}

    \# & $\;$A & \multicolumn{4}{c}{Game Master} & $\;\:$B\\
    \hline 

    \theutterance \stepcounter{utterance}  

    & & \multicolumn{4}{p{0.6\linewidth}}{\cellcolor[rgb]{0.9,0.9,0.9}{%
	\makecell[{{p{\linewidth}}}]{% 
	  \tt {\tiny [A$\langle$GM]}  
	 You are a language wizard who likes to guess words by using the given rules.\\ \tt \\ \tt Welcome to Wordle! You have six attempts to guess the target word, a valid English word of five lowercase letters (a-z). Please use the tags "guess:" and "explanation:" to provide a concise explanation for each guess.\\ \tt \\ \tt To help you make an informed guess, you will receive a clue for the word, such as\\ \tt clue: "snowy white"\\ \tt \\ \tt Here is an example guess based on the clue:\\ \tt guess: apple\\ \tt explanation: In the fairy tail Snow White, the girl is killed because she eats a poisoned apple. And the word apple has 5 letters.\\ \tt \\ \tt I will then indicate whether I agree or disagree with your guess and provide rationale, but agreeing with a guess does not confirm its correctness. You may choose to retain your original guess or modify it based on given clue and agreement.\\ \tt \\ \tt After each guess, your answer will be validated, and you will receive feedback indicating which letters are correct (green), which letters are correct but in the wrong position (yellow), and which letters are incorrect (red). This feedback can be useful in determining which letters to include or exclude in your next guess.\\ \tt \\ \tt For example, the feedback for "apple" might be:\\ \tt guess_feedback: a<yellow> p<yellow> p<green> l<yellow> e<red>\\ \tt \\ \tt Here are some rules to keep in mind:\\ \tt 1. Please do not generate any guess feedback or clue on your own\\ \tt 2. Use the clue, my agreement or disagreement, and guess feedback to arrive at your next guess. This should be reflected in your explanation as well.\\ \tt \\ \tt Let's begin with your first guess.\\ \tt \\ \tt clue:hungry\\ \tt  
	  } 
	   } 
	   } 
	 & \\ 
 

    \theutterance \stepcounter{utterance}  

    & \multicolumn{4}{p{0.6\linewidth}}{\cellcolor[rgb]{0.8,1,0.9}{%
	\makecell[{{p{\linewidth}}}]{% 
	  \tt {\tiny [A$\rangle$GM]}  
	 guess: apple\\ \tt explanation: The word "apple" is often associated with the phrase "the apple a day keeps the doctor away," which can be interpreted as a reminder to eat healthy, including fruits like apples. Additionally, the word "apple" has five letters, fitting the criteria for the Wordle game. 
	  } 
	   } 
	   } 
	 & & \\ 
 

    \theutterance \stepcounter{utterance}  

    & & & \multicolumn{2}{p{0.3\linewidth}}{\cellcolor[rgb]{0.95,0.95,0.95}{%
	\makecell[{{p{\linewidth}}}]{% 
	  \tt {\tiny [GM$|$GM]}  
	 {'guess': 'apple', 'explanation': 'The word "apple" is often associated with the phrase "the apple a day keeps the doctor away," which can be interpreted as a reminder to eat healthy, including fruits like apples. Additionally, the word "apple" has five letters, fitting the criteria for the Wordle game.'} 
	  } 
	   } 
	   } 
	 & & \\ 
 

    \theutterance \stepcounter{utterance}  

    & & \multicolumn{4}{p{0.6\linewidth}}{\cellcolor[rgb]{0.9,0.9,0.9}{%
	\makecell[{{p{\linewidth}}}]{% 
	  \tt {\tiny [GM$\rangle$B]}  
	 I need your assistance with a word game in which we need to find a 5-letter word using a clue, a guess and an explanation for the guess.\\ \tt \\ \tt For example, if the clue is "snow white" and the guess is "apple", you will get the information like this:\\ \tt clue:snow white\\ \tt guess:apple\\ \tt explanation:the fairy tale Snow White involves the girl eating a poisoned apple, which has 5 letters.\\ \tt \\ \tt Your task is to either agree or disagree with my guess based on the given clue.\\ \tt \\ \tt If you disagree, I will make another guess, and if you agree, I will proceed in the game using the same guess.\\ \tt \\ \tt Initially, guess feedback is not available, but as the game proceeds, it becomes available. The guess feedback is not for current guess, but rather an indication of what letters can be present in the current guess. A letter highlighted in green indicates that it is a correct letter in the correct position, while yellow indicates that it is a correct letter in the wrong position, and red indicates a wrong letter. At this point, you should use all the provided information, including the clue, guess, explanation, and guess feedback, to decide whether the given guess matches the clue, and also stick to guess feedback or not.\\ \tt \\ \tt Please respond in lowercase letters and stick to this format:\\ \tt agreement:yes or no\\ \tt explanation:your reason for doing so\\ \tt \\ \tt Please note that you must not guess any word based on the riddle clue and stick to the given format while responding.\\ \tt \\ \tt Let's begin!\\ \tt \\ \tt clue:hungry\\ \tt guess:apple\\ \tt explanation:The word "apple" is often associated with the phrase "the apple a day keeps the doctor away," which can be interpreted as a reminder to eat healthy, including fruits like apples. Additionally, the word "apple" has five letters, fitting the criteria for the Wordle game.\\ \tt  
	  } 
	   } 
	   } 
	 & \\ 
 

    \theutterance \stepcounter{utterance}  

    & & & \multicolumn{4}{p{0.6\linewidth}}{\cellcolor[rgb]{1,0.85,0.72}{%
	\makecell[{{p{\linewidth}}}]{% 
	  \tt {\tiny [GM$\langle$B]}  
	 agreement:no\\ \tt explanation:Although the word "apple" has five letters and is a fruit, it does not directly relate to the clue "hungry." The phrase "the apple a day keeps the doctor away" is a common English proverb, but it does not convey the idea of hunger. A more suitable guess for the clue "hungry" would be a word that directly relates to the concept of hunger or thirst. 
	  } 
	   } 
	   } 
	  \\ 
 

    \theutterance \stepcounter{utterance}  

    & & & \multicolumn{2}{p{0.3\linewidth}}{\cellcolor[rgb]{0.95,0.95,0.95}{%
	\makecell[{{p{\linewidth}}}]{% 
	  \tt {\tiny [GM$|$GM]}  
	 {'agreement': 'no', 'explanation': 'Although the word "apple" has five letters and is a fruit, it does not directly relate to the clue "hungry." The phrase "the apple a day keeps the doctor away" is a common English proverb, but it does not convey the idea of hunger. A more suitable guess for the clue "hungry" would be a word that directly relates to the concept of hunger or thirst.'} 
	  } 
	   } 
	   } 
	 & & \\ 
 

    \theutterance \stepcounter{utterance}  

    & & & \multicolumn{2}{p{0.3\linewidth}}{\cellcolor[rgb]{0.95,0.95,0.95}{%
	\makecell[{{p{\linewidth}}}]{% 
	  \tt {\tiny [GM$|$GM]}  
	 Critic disagrees with the Guesser -- Sharing the critic's explanation with the guesser 
	  } 
	   } 
	   } 
	 & & \\ 
 

    \theutterance \stepcounter{utterance}  

    & & \multicolumn{4}{p{0.6\linewidth}}{\cellcolor[rgb]{0.9,0.9,0.9}{%
	\makecell[{{p{\linewidth}}}]{% 
	  \tt {\tiny [A$\langle$GM]}  
	 clue:hungry\\ \tt guess_agreement:no\\ \tt agreement_explanation:Although the word "apple" has five letters and is a fruit, it does not directly relate to the clue "hungry." The phrase "the apple a day keeps the doctor away" is a common English proverb, but it does not convey the idea of hunger. A more suitable guess for the clue "hungry" would be a word that directly relates to the concept of hunger or thirst.\\ \tt  
	  } 
	   } 
	   } 
	 & \\ 
 

    \theutterance \stepcounter{utterance}  

    & \multicolumn{4}{p{0.6\linewidth}}{\cellcolor[rgb]{0.8,1,0.9}{%
	\makecell[{{p{\linewidth}}}]{% 
	  \tt {\tiny [A$\rangle$GM]}  
	 guess:grape\\ \tt explanation:The word "grape" is a type of fruit that has five letters. It is often associated with the phrase "grape expectations," which can be interpreted as having high hopes or expectations. The word "grape" also shares some similarities with the word "crave," which means to have a strong desire for something. This can be related to the concept of hunger, as craving food is a common experience for many people. 
	  } 
	   } 
	   } 
	 & & \\ 
 

    \theutterance \stepcounter{utterance}  

    & & & \multicolumn{2}{p{0.3\linewidth}}{\cellcolor[rgb]{0.95,0.95,0.95}{%
	\makecell[{{p{\linewidth}}}]{% 
	  \tt {\tiny [GM$|$GM]}  
	 {'guess': 'grape', 'explanation': 'The word "grape" is a type of fruit that has five letters. It is often associated with the phrase "grape expectations," which can be interpreted as having high hopes or expectations. The word "grape" also shares some similarities with the word "crave," which means to have a strong desire for something. This can be related to the concept of hunger, as craving food is a common experience for many people.'} 
	  } 
	   } 
	   } 
	 & & \\ 
 

    \theutterance \stepcounter{utterance}  

    & & & \multicolumn{2}{p{0.3\linewidth}}{\cellcolor[rgb]{0.95,0.95,0.95}{%
	\makecell[{{p{\linewidth}}}]{% 
	  \tt {\tiny [GM$|$GM]}  
	 Change in player1's guess\\ \tt guess_before_critic_opinion: apple\\ \tt                                                 critic_agreement: no\\ \tt guess_after_critic_opinion: grape\\ \tt                                                 Proceeding with guess validation 
	  } 
	   } 
	   } 
	 & & \\ 
 

    \theutterance \stepcounter{utterance}  

    & & & \multicolumn{2}{p{0.3\linewidth}}{\cellcolor[rgb]{0.95,0.95,0.95}{%
	\makecell[{{p{\linewidth}}}]{% 
	  \tt {\tiny [GM$|$GM]}  
	 attempts: 1\\ \tt target_word = unfed\\ \tt target_word_clue = hungry\\ \tt guess: grape\\ \tt guess_feedback: g<red> r<red> a<red> p<red> e<yellow> 
	  } 
	   } 
	   } 
	 & & \\ 
 

    \theutterance \stepcounter{utterance}  

    & & \multicolumn{4}{p{0.6\linewidth}}{\cellcolor[rgb]{0.9,0.9,0.9}{%
	\makecell[{{p{\linewidth}}}]{% 
	  \tt {\tiny [A$\langle$GM]}  
	 guess_feedback: g<red> r<red> a<red> p<red> e<yellow> 
	  } 
	   } 
	   } 
	 & \\ 
 

    \theutterance \stepcounter{utterance}  

    & \multicolumn{4}{p{0.6\linewidth}}{\cellcolor[rgb]{0.8,1,0.9}{%
	\makecell[{{p{\linewidth}}}]{% 
	  \tt {\tiny [A$\rangle$GM]}  
	 guess:peach\\ \tt explanation:The feedback from the previous guess indicates that the letter "e" is in the wrong position. The word "peach" is a type of fruit that has five letters and shares the letter "e" with the word "grape." It also has a similar sound and structure to the word "crave," which can be related to the concept of hunger. 
	  } 
	   } 
	   } 
	 & & \\ 
 

    \theutterance \stepcounter{utterance}  

    & & & \multicolumn{2}{p{0.3\linewidth}}{\cellcolor[rgb]{0.95,0.95,0.95}{%
	\makecell[{{p{\linewidth}}}]{% 
	  \tt {\tiny [GM$|$GM]}  
	 {'guess': 'peach', 'explanation': 'The feedback from the previous guess indicates that the letter "e" is in the wrong position. The word "peach" is a type of fruit that has five letters and shares the letter "e" with the word "grape." It also has a similar sound and structure to the word "crave," which can be related to the concept of hunger.'} 
	  } 
	   } 
	   } 
	 & & \\ 
 

    \theutterance \stepcounter{utterance}  

    & & \multicolumn{4}{p{0.6\linewidth}}{\cellcolor[rgb]{0.9,0.9,0.9}{%
	\makecell[{{p{\linewidth}}}]{% 
	  \tt {\tiny [GM$\rangle$B]}  
	 clue:hungry\\ \tt guess:peach\\ \tt explanation:The feedback from the previous guess indicates that the letter "e" is in the wrong position. The word "peach" is a type of fruit that has five letters and shares the letter "e" with the word "grape." It also has a similar sound and structure to the word "crave," which can be related to the concept of hunger.\\ \tt guess_feedback: g<red> r<red> a<red> p<red> e<yellow> 
	  } 
	   } 
	   } 
	 & \\ 
 

    \theutterance \stepcounter{utterance}  

    & & & \multicolumn{4}{p{0.6\linewidth}}{\cellcolor[rgb]{1,0.85,0.72}{%
	\makecell[{{p{\linewidth}}}]{% 
	  \tt {\tiny [GM$\langle$B]}  
	 agreement:no\\ \tt explanation:Although the word "peach" has the correct number of letters and shares some similarities with the word "crave," it does not directly relate to the concept of hunger. The guess feedback also indicates that the letter "e" is in the wrong position, which further suggests that "peach" is not the correct answer. 
	  } 
	   } 
	   } 
	  \\ 
 

    \theutterance \stepcounter{utterance}  

    & & & \multicolumn{2}{p{0.3\linewidth}}{\cellcolor[rgb]{0.95,0.95,0.95}{%
	\makecell[{{p{\linewidth}}}]{% 
	  \tt {\tiny [GM$|$GM]}  
	 {'agreement': 'no', 'explanation': 'Although the word "peach" has the correct number of letters and shares some similarities with the word "crave," it does not directly relate to the concept of hunger. The guess feedback also indicates that the letter "e" is in the wrong position, which further suggests that "peach" is not the correct answer.'} 
	  } 
	   } 
	   } 
	 & & \\ 
 

    \theutterance \stepcounter{utterance}  

    & & & \multicolumn{2}{p{0.3\linewidth}}{\cellcolor[rgb]{0.95,0.95,0.95}{%
	\makecell[{{p{\linewidth}}}]{% 
	  \tt {\tiny [GM$|$GM]}  
	 Critic disagrees with the Guesser -- Sharing the critic's explanation with the guesser 
	  } 
	   } 
	   } 
	 & & \\ 
 

    \theutterance \stepcounter{utterance}  

    & & \multicolumn{4}{p{0.6\linewidth}}{\cellcolor[rgb]{0.9,0.9,0.9}{%
	\makecell[{{p{\linewidth}}}]{% 
	  \tt {\tiny [A$\langle$GM]}  
	 clue:hungry\\ \tt guess_agreement:no\\ \tt agreement_explanation:Although the word "peach" has the correct number of letters and shares some similarities with the word "crave," it does not directly relate to the concept of hunger. The guess feedback also indicates that the letter "e" is in the wrong position, which further suggests that "peach" is not the correct answer.\\ \tt  
	  } 
	   } 
	   } 
	 & \\ 
 

    \theutterance \stepcounter{utterance}  

    & \multicolumn{4}{p{0.6\linewidth}}{\cellcolor[rgb]{0.8,1,0.9}{%
	\makecell[{{p{\linewidth}}}]{% 
	  \tt {\tiny [A$\rangle$GM]}  
	 guess:crave\\ \tt explanation:The feedback from the previous guess indicates that the letter "e" is in the wrong position. The word "crave" is a verb that means to have a strong desire for something, which can be directly related to the concept of hunger. It also shares the letter "e" with the word "grape" and has a similar sound and structure to the word "peach." 
	  } 
	   } 
	   } 
	 & & \\ 
 

    \theutterance \stepcounter{utterance}  

    & & & \multicolumn{2}{p{0.3\linewidth}}{\cellcolor[rgb]{0.95,0.95,0.95}{%
	\makecell[{{p{\linewidth}}}]{% 
	  \tt {\tiny [GM$|$GM]}  
	 {'guess': 'crave', 'explanation': 'The feedback from the previous guess indicates that the letter "e" is in the wrong position. The word "crave" is a verb that means to have a strong desire for something, which can be directly related to the concept of hunger. It also shares the letter "e" with the word "grape" and has a similar sound and structure to the word "peach."'} 
	  } 
	   } 
	   } 
	 & & \\ 
 

    \theutterance \stepcounter{utterance}  

    & & & \multicolumn{2}{p{0.3\linewidth}}{\cellcolor[rgb]{0.95,0.95,0.95}{%
	\makecell[{{p{\linewidth}}}]{% 
	  \tt {\tiny [GM$|$GM]}  
	 Change in player1's guess\\ \tt guess_before_critic_opinion: peach\\ \tt                                                 critic_agreement: no\\ \tt guess_after_critic_opinion: crave\\ \tt                                                 Proceeding with guess validation 
	  } 
	   } 
	   } 
	 & & \\ 
 

    \theutterance \stepcounter{utterance}  

    & & & \multicolumn{2}{p{0.3\linewidth}}{\cellcolor[rgb]{0.95,0.95,0.95}{%
	\makecell[{{p{\linewidth}}}]{% 
	  \tt {\tiny [GM$|$GM]}  
	 attempts: 2\\ \tt target_word = unfed\\ \tt target_word_clue = hungry\\ \tt guess: crave\\ \tt guess_feedback: c<red> r<red> a<red> v<red> e<yellow> 
	  } 
	   } 
	   } 
	 & & \\ 
 

    \theutterance \stepcounter{utterance}  

    & & \multicolumn{4}{p{0.6\linewidth}}{\cellcolor[rgb]{0.9,0.9,0.9}{%
	\makecell[{{p{\linewidth}}}]{% 
	  \tt {\tiny [A$\langle$GM]}  
	 guess_feedback: c<red> r<red> a<red> v<red> e<yellow> 
	  } 
	   } 
	   } 
	 & \\ 
 

    \theutterance \stepcounter{utterance}  

    & \multicolumn{4}{p{0.6\linewidth}}{\cellcolor[rgb]{0.8,1,0.9}{%
	\makecell[{{p{\linewidth}}}]{% 
	  \tt {\tiny [A$\rangle$GM]}  
	 guess:deify\\ \tt explanation:The feedback from the previous guess indicates that the letter "e" is in the wrong position. The word "deify" is a verb that means to treat someone as a god or to give them excessive praise. While it may not seem directly related to the concept of hunger, it shares the letter "e" with the word "crave" and has a similar sound and structure to the word "peach." 
	  } 
	   } 
	   } 
	 & & \\ 
 

    \theutterance \stepcounter{utterance}  

    & & & \multicolumn{2}{p{0.3\linewidth}}{\cellcolor[rgb]{0.95,0.95,0.95}{%
	\makecell[{{p{\linewidth}}}]{% 
	  \tt {\tiny [GM$|$GM]}  
	 {'guess': 'deify', 'explanation': 'The feedback from the previous guess indicates that the letter "e" is in the wrong position. The word "deify" is a verb that means to treat someone as a god or to give them excessive praise. While it may not seem directly related to the concept of hunger, it shares the letter "e" with the word "crave" and has a similar sound and structure to the word "peach."'} 
	  } 
	   } 
	   } 
	 & & \\ 
 

    \theutterance \stepcounter{utterance}  

    & & \multicolumn{4}{p{0.6\linewidth}}{\cellcolor[rgb]{0.9,0.9,0.9}{%
	\makecell[{{p{\linewidth}}}]{% 
	  \tt {\tiny [GM$\rangle$B]}  
	 clue:hungry\\ \tt guess:deify\\ \tt explanation:The feedback from the previous guess indicates that the letter "e" is in the wrong position. The word "deify" is a verb that means to treat someone as a god or to give them excessive praise. While it may not seem directly related to the concept of hunger, it shares the letter "e" with the word "crave" and has a similar sound and structure to the word "peach."\\ \tt guess_feedback: c<red> r<red> a<red> v<red> e<yellow> 
	  } 
	   } 
	   } 
	 & \\ 
 

    \theutterance \stepcounter{utterance}  

    & & & \multicolumn{4}{p{0.6\linewidth}}{\cellcolor[rgb]{1,0.85,0.72}{%
	\makecell[{{p{\linewidth}}}]{% 
	  \tt {\tiny [GM$\langle$B]}  
	 agreement:no\\ \tt explanation:Although the word "deify" shares the letter "e" with the word "crave" and has a similar sound and structure to the word "peach," it does not seem to be a good fit for the clue "hungry." The word "deify" is a verb that means to treat someone as a god or to give them excessive praise, which does not appear to be related to the concept of hunger. 
	  } 
	   } 
	   } 
	  \\ 
 

    \theutterance \stepcounter{utterance}  

    & & & \multicolumn{2}{p{0.3\linewidth}}{\cellcolor[rgb]{0.95,0.95,0.95}{%
	\makecell[{{p{\linewidth}}}]{% 
	  \tt {\tiny [GM$|$GM]}  
	 {'agreement': 'no', 'explanation': 'Although the word "deify" shares the letter "e" with the word "crave" and has a similar sound and structure to the word "peach," it does not seem to be a good fit for the clue "hungry." The word "deify" is a verb that means to treat someone as a god or to give them excessive praise, which does not appear to be related to the concept of hunger.'} 
	  } 
	   } 
	   } 
	 & & \\ 
 

    \theutterance \stepcounter{utterance}  

    & & & \multicolumn{2}{p{0.3\linewidth}}{\cellcolor[rgb]{0.95,0.95,0.95}{%
	\makecell[{{p{\linewidth}}}]{% 
	  \tt {\tiny [GM$|$GM]}  
	 Critic disagrees with the Guesser -- Sharing the critic's explanation with the guesser 
	  } 
	   } 
	   } 
	 & & \\ 
 

    \theutterance \stepcounter{utterance}  

    & & \multicolumn{4}{p{0.6\linewidth}}{\cellcolor[rgb]{0.9,0.9,0.9}{%
	\makecell[{{p{\linewidth}}}]{% 
	  \tt {\tiny [A$\langle$GM]}  
	 clue:hungry\\ \tt guess_agreement:no\\ \tt agreement_explanation:Although the word "deify" shares the letter "e" with the word "crave" and has a similar sound and structure to the word "peach," it does not seem to be a good fit for the clue "hungry." The word "deify" is a verb that means to treat someone as a god or to give them excessive praise, which does not appear to be related to the concept of hunger.\\ \tt  
	  } 
	   } 
	   } 
	 & \\ 
 

    \theutterance \stepcounter{utterance}  

    & \multicolumn{4}{p{0.6\linewidth}}{\cellcolor[rgb]{0.8,1,0.9}{%
	\makecell[{{p{\linewidth}}}]{% 
	  \tt {\tiny [A$\rangle$GM]}  
	 guess:gluey\\ \tt explanation:The feedback from the previous guess indicates that the letter "e" is in the wrong position. The word "gluey" is an adjective that means sticky or tacky. While it may not seem directly related to the concept of hunger, it shares the letter "e" with the word "crave" and has a similar sound and structure to the word "peach." 
	  } 
	   } 
	   } 
	 & & \\ 
 

    \theutterance \stepcounter{utterance}  

    & & & \multicolumn{2}{p{0.3\linewidth}}{\cellcolor[rgb]{0.95,0.95,0.95}{%
	\makecell[{{p{\linewidth}}}]{% 
	  \tt {\tiny [GM$|$GM]}  
	 {'guess': 'gluey', 'explanation': 'The feedback from the previous guess indicates that the letter "e" is in the wrong position. The word "gluey" is an adjective that means sticky or tacky. While it may not seem directly related to the concept of hunger, it shares the letter "e" with the word "crave" and has a similar sound and structure to the word "peach."'} 
	  } 
	   } 
	   } 
	 & & \\ 
 

    \theutterance \stepcounter{utterance}  

    & & & \multicolumn{2}{p{0.3\linewidth}}{\cellcolor[rgb]{0.95,0.95,0.95}{%
	\makecell[{{p{\linewidth}}}]{% 
	  \tt {\tiny [GM$|$GM]}  
	 Change in player1's guess\\ \tt guess_before_critic_opinion: deify\\ \tt                                                 critic_agreement: no\\ \tt guess_after_critic_opinion: gluey\\ \tt                                                 Proceeding with guess validation 
	  } 
	   } 
	   } 
	 & & \\ 
 

    \theutterance \stepcounter{utterance}  

    & & & \multicolumn{2}{p{0.3\linewidth}}{\cellcolor[rgb]{0.95,0.95,0.95}{%
	\makecell[{{p{\linewidth}}}]{% 
	  \tt {\tiny [GM$|$GM]}  
	 attempts: 3\\ \tt target_word = unfed\\ \tt target_word_clue = hungry\\ \tt guess: gluey\\ \tt guess_feedback: g<red> l<red> u<yellow> e<green> y<red> 
	  } 
	   } 
	   } 
	 & & \\ 
 

    \theutterance \stepcounter{utterance}  

    & & \multicolumn{4}{p{0.6\linewidth}}{\cellcolor[rgb]{0.9,0.9,0.9}{%
	\makecell[{{p{\linewidth}}}]{% 
	  \tt {\tiny [A$\langle$GM]}  
	 guess_feedback: g<red> l<red> u<yellow> e<green> y<red> 
	  } 
	   } 
	   } 
	 & \\ 
 

    \theutterance \stepcounter{utterance}  

    & \multicolumn{4}{p{0.6\linewidth}}{\cellcolor[rgb]{0.8,1,0.9}{%
	\makecell[{{p{\linewidth}}}]{% 
	  \tt {\tiny [A$\rangle$GM]}  
	 guess:gluey\\ \tt explanation:The feedback from the previous guess indicates that the letter "e" is in the correct position and the letter "u" is in the wrong position. The word "gluey" is an adjective that means sticky or tacky. While it may not seem directly related to the concept of hunger, it shares the letter "e" with the word "crave" and has a similar sound and structure to the word "peach." 
	  } 
	   } 
	   } 
	 & & \\ 
 

    \theutterance \stepcounter{utterance}  

    & & & \multicolumn{2}{p{0.3\linewidth}}{\cellcolor[rgb]{0.95,0.95,0.95}{%
	\makecell[{{p{\linewidth}}}]{% 
	  \tt {\tiny [GM$|$GM]}  
	 {'guess': 'gluey', 'explanation': 'The feedback from the previous guess indicates that the letter "e" is in the correct position and the letter "u" is in the wrong position. The word "gluey" is an adjective that means sticky or tacky. While it may not seem directly related to the concept of hunger, it shares the letter "e" with the word "crave" and has a similar sound and structure to the word "peach."'} 
	  } 
	   } 
	   } 
	 & & \\ 
 

    \theutterance \stepcounter{utterance}  

    & & \multicolumn{4}{p{0.6\linewidth}}{\cellcolor[rgb]{0.9,0.9,0.9}{%
	\makecell[{{p{\linewidth}}}]{% 
	  \tt {\tiny [GM$\rangle$B]}  
	 clue:hungry\\ \tt guess:gluey\\ \tt explanation:The feedback from the previous guess indicates that the letter "e" is in the correct position and the letter "u" is in the wrong position. The word "gluey" is an adjective that means sticky or tacky. While it may not seem directly related to the concept of hunger, it shares the letter "e" with the word "crave" and has a similar sound and structure to the word "peach."\\ \tt guess_feedback: g<red> l<red> u<yellow> e<green> y<red> 
	  } 
	   } 
	   } 
	 & \\ 
 

    \theutterance \stepcounter{utterance}  

    & & & \multicolumn{4}{p{0.6\linewidth}}{\cellcolor[rgb]{1,0.85,0.72}{%
	\makecell[{{p{\linewidth}}}]{% 
	  \tt {\tiny [GM$\langle$B]}  
	 agreement:no\\ \tt explanation:Although the word "gluey" shares the letter "e" with the word "crave" and has a similar sound and structure to the word "peach," it does not seem to be a good fit for the clue "hungry." The word "gluey" is an adjective that describes a sticky or tacky substance, which does not appear to be related to the concept of hunger. 
	  } 
	   } 
	   } 
	  \\ 
 

    \theutterance \stepcounter{utterance}  

    & & & \multicolumn{2}{p{0.3\linewidth}}{\cellcolor[rgb]{0.95,0.95,0.95}{%
	\makecell[{{p{\linewidth}}}]{% 
	  \tt {\tiny [GM$|$GM]}  
	 {'agreement': 'no', 'explanation': 'Although the word "gluey" shares the letter "e" with the word "crave" and has a similar sound and structure to the word "peach," it does not seem to be a good fit for the clue "hungry." The word "gluey" is an adjective that describes a sticky or tacky substance, which does not appear to be related to the concept of hunger.'} 
	  } 
	   } 
	   } 
	 & & \\ 
 

    \theutterance \stepcounter{utterance}  

    & & & \multicolumn{2}{p{0.3\linewidth}}{\cellcolor[rgb]{0.95,0.95,0.95}{%
	\makecell[{{p{\linewidth}}}]{% 
	  \tt {\tiny [GM$|$GM]}  
	 Critic disagrees with the Guesser -- Sharing the critic's explanation with the guesser 
	  } 
	   } 
	   } 
	 & & \\ 
 

    \theutterance \stepcounter{utterance}  

    & & \multicolumn{4}{p{0.6\linewidth}}{\cellcolor[rgb]{0.9,0.9,0.9}{%
	\makecell[{{p{\linewidth}}}]{% 
	  \tt {\tiny [A$\langle$GM]}  
	 clue:hungry\\ \tt guess_agreement:no\\ \tt agreement_explanation:Although the word "gluey" shares the letter "e" with the word "crave" and has a similar sound and structure to the word "peach," it does not seem to be a good fit for the clue "hungry." The word "gluey" is an adjective that describes a sticky or tacky substance, which does not appear to be related to the concept of hunger.\\ \tt  
	  } 
	   } 
	   } 
	 & \\ 
 

    \theutterance \stepcounter{utterance}  

    & \multicolumn{4}{p{0.6\linewidth}}{\cellcolor[rgb]{0.8,1,0.9}{%
	\makecell[{{p{\linewidth}}}]{% 
	  \tt {\tiny [A$\rangle$GM]}  
	 guess:grape\\ \tt explanation:The feedback from the previous guess indicates that the letter "e" is in the correct position and the letter "u" is in the wrong position. The word "grape" is a type of fruit that has five letters and shares the letter "e" with the word "crave." It also has a similar sound and structure to the word "peach." 
	  } 
	   } 
	   } 
	 & & \\ 
 

    \theutterance \stepcounter{utterance}  

    & & & \multicolumn{2}{p{0.3\linewidth}}{\cellcolor[rgb]{0.95,0.95,0.95}{%
	\makecell[{{p{\linewidth}}}]{% 
	  \tt {\tiny [GM$|$GM]}  
	 {'guess': 'grape', 'explanation': 'The feedback from the previous guess indicates that the letter "e" is in the correct position and the letter "u" is in the wrong position. The word "grape" is a type of fruit that has five letters and shares the letter "e" with the word "crave." It also has a similar sound and structure to the word "peach."'} 
	  } 
	   } 
	   } 
	 & & \\ 
 

    \theutterance \stepcounter{utterance}  

    & & & \multicolumn{2}{p{0.3\linewidth}}{\cellcolor[rgb]{0.95,0.95,0.95}{%
	\makecell[{{p{\linewidth}}}]{% 
	  \tt {\tiny [GM$|$GM]}  
	 Change in player1's guess\\ \tt guess_before_critic_opinion: gluey\\ \tt                                                 critic_agreement: no\\ \tt guess_after_critic_opinion: grape\\ \tt                                                 Proceeding with guess validation 
	  } 
	   } 
	   } 
	 & & \\ 
 

    \theutterance \stepcounter{utterance}  

    & & & \multicolumn{2}{p{0.3\linewidth}}{\cellcolor[rgb]{0.95,0.95,0.95}{%
	\makecell[{{p{\linewidth}}}]{% 
	  \tt {\tiny [GM$|$GM]}  
	 attempts: 4\\ \tt target_word = unfed\\ \tt target_word_clue = hungry\\ \tt guess: grape\\ \tt guess_feedback: g<red> r<red> a<red> p<red> e<yellow> 
	  } 
	   } 
	   } 
	 & & \\ 
 

    \theutterance \stepcounter{utterance}  

    & & \multicolumn{4}{p{0.6\linewidth}}{\cellcolor[rgb]{0.9,0.9,0.9}{%
	\makecell[{{p{\linewidth}}}]{% 
	  \tt {\tiny [A$\langle$GM]}  
	 guess_feedback: g<red> r<red> a<red> p<red> e<yellow> 
	  } 
	   } 
	   } 
	 & \\ 
 

    \theutterance \stepcounter{utterance}  

    & \multicolumn{4}{p{0.6\linewidth}}{\cellcolor[rgb]{0.8,1,0.9}{%
	\makecell[{{p{\linewidth}}}]{% 
	  \tt {\tiny [A$\rangle$GM]}  
	 guess:creed\\ \tt explanation:The feedback from the previous guess indicates that the letter "e" is in the wrong position. The word "creed" is a noun that means a set of principles or beliefs. While it may not seem directly related to the concept of hunger, it shares the letter "e" with the word "grape" and has a similar sound and structure to the word "peach." 
	  } 
	   } 
	   } 
	 & & \\ 
 

    \theutterance \stepcounter{utterance}  

    & & & \multicolumn{2}{p{0.3\linewidth}}{\cellcolor[rgb]{0.95,0.95,0.95}{%
	\makecell[{{p{\linewidth}}}]{% 
	  \tt {\tiny [GM$|$GM]}  
	 {'guess': 'creed', 'explanation': 'The feedback from the previous guess indicates that the letter "e" is in the wrong position. The word "creed" is a noun that means a set of principles or beliefs. While it may not seem directly related to the concept of hunger, it shares the letter "e" with the word "grape" and has a similar sound and structure to the word "peach."'} 
	  } 
	   } 
	   } 
	 & & \\ 
 

    \theutterance \stepcounter{utterance}  

    & & \multicolumn{4}{p{0.6\linewidth}}{\cellcolor[rgb]{0.9,0.9,0.9}{%
	\makecell[{{p{\linewidth}}}]{% 
	  \tt {\tiny [GM$\rangle$B]}  
	 clue:hungry\\ \tt guess:creed\\ \tt explanation:The feedback from the previous guess indicates that the letter "e" is in the wrong position. The word "creed" is a noun that means a set of principles or beliefs. While it may not seem directly related to the concept of hunger, it shares the letter "e" with the word "grape" and has a similar sound and structure to the word "peach."\\ \tt guess_feedback: g<red> r<red> a<red> p<red> e<yellow> 
	  } 
	   } 
	   } 
	 & \\ 
 

    \theutterance \stepcounter{utterance}  

    & & & \multicolumn{4}{p{0.6\linewidth}}{\cellcolor[rgb]{1,0.85,0.72}{%
	\makecell[{{p{\linewidth}}}]{% 
	  \tt {\tiny [GM$\langle$B]}  
	 agreement:no\\ \tt explanation:Although the word "creed" shares the letter "e" with the word "grape" and has a similar sound and structure to the word "peach," it does not seem to be a good fit for the clue "hungry." The word "creed" does not have any obvious connections to the concept of hunger, and the guess feedback does not provide any evidence to suggest that the word "creed" could be a possible solution. 
	  } 
	   } 
	   } 
	  \\ 
 

    \theutterance \stepcounter{utterance}  

    & & & \multicolumn{2}{p{0.3\linewidth}}{\cellcolor[rgb]{0.95,0.95,0.95}{%
	\makecell[{{p{\linewidth}}}]{% 
	  \tt {\tiny [GM$|$GM]}  
	 {'agreement': 'no', 'explanation': 'Although the word "creed" shares the letter "e" with the word "grape" and has a similar sound and structure to the word "peach," it does not seem to be a good fit for the clue "hungry." The word "creed" does not have any obvious connections to the concept of hunger, and the guess feedback does not provide any evidence to suggest that the word "creed" could be a possible solution.'} 
	  } 
	   } 
	   } 
	 & & \\ 
 

    \theutterance \stepcounter{utterance}  

    & & & \multicolumn{2}{p{0.3\linewidth}}{\cellcolor[rgb]{0.95,0.95,0.95}{%
	\makecell[{{p{\linewidth}}}]{% 
	  \tt {\tiny [GM$|$GM]}  
	 Critic disagrees with the Guesser -- Sharing the critic's explanation with the guesser 
	  } 
	   } 
	   } 
	 & & \\ 
 

    \theutterance \stepcounter{utterance}  

    & & \multicolumn{4}{p{0.6\linewidth}}{\cellcolor[rgb]{0.9,0.9,0.9}{%
	\makecell[{{p{\linewidth}}}]{% 
	  \tt {\tiny [A$\langle$GM]}  
	 clue:hungry\\ \tt guess_agreement:no\\ \tt agreement_explanation:Although the word "creed" shares the letter "e" with the word "grape" and has a similar sound and structure to the word "peach," it does not seem to be a good fit for the clue "hungry." The word "creed" does not have any obvious connections to the concept of hunger, and the guess feedback does not provide any evidence to suggest that the word "creed" could be a possible solution.\\ \tt  
	  } 
	   } 
	   } 
	 & \\ 
 

    \theutterance \stepcounter{utterance}  

    & \multicolumn{4}{p{0.6\linewidth}}{\cellcolor[rgb]{0.8,1,0.9}{%
	\makecell[{{p{\linewidth}}}]{% 
	  \tt {\tiny [A$\rangle$GM]}  
	 guess:crave\\ \tt explanation:The feedback from the previous guess indicates that the letter "e" is in the correct position and the letter "u" is in the wrong position. The word "crave" is a verb that means to have a strong desire for something, which can be directly related to the concept of hunger. It also shares the letter "e" with the word "grape" and has a similar sound and structure to the word "peach." 
	  } 
	   } 
	   } 
	 & & \\ 
 

    \theutterance \stepcounter{utterance}  

    & & & \multicolumn{2}{p{0.3\linewidth}}{\cellcolor[rgb]{0.95,0.95,0.95}{%
	\makecell[{{p{\linewidth}}}]{% 
	  \tt {\tiny [GM$|$GM]}  
	 {'guess': 'crave', 'explanation': 'The feedback from the previous guess indicates that the letter "e" is in the correct position and the letter "u" is in the wrong position. The word "crave" is a verb that means to have a strong desire for something, which can be directly related to the concept of hunger. It also shares the letter "e" with the word "grape" and has a similar sound and structure to the word "peach."'} 
	  } 
	   } 
	   } 
	 & & \\ 
 

    \theutterance \stepcounter{utterance}  

    & & & \multicolumn{2}{p{0.3\linewidth}}{\cellcolor[rgb]{0.95,0.95,0.95}{%
	\makecell[{{p{\linewidth}}}]{% 
	  \tt {\tiny [GM$|$GM]}  
	 Change in player1's guess\\ \tt guess_before_critic_opinion: creed\\ \tt                                                 critic_agreement: no\\ \tt guess_after_critic_opinion: crave\\ \tt                                                 Proceeding with guess validation 
	  } 
	   } 
	   } 
	 & & \\ 
 

    \theutterance \stepcounter{utterance}  

    & & & \multicolumn{2}{p{0.3\linewidth}}{\cellcolor[rgb]{0.95,0.95,0.95}{%
	\makecell[{{p{\linewidth}}}]{% 
	  \tt {\tiny [GM$|$GM]}  
	 attempts: 5\\ \tt target_word = unfed\\ \tt target_word_clue = hungry\\ \tt guess: crave\\ \tt guess_feedback: c<red> r<red> a<red> v<red> e<yellow> 
	  } 
	   } 
	   } 
	 & & \\ 
 

    \theutterance \stepcounter{utterance}  

    & & \multicolumn{4}{p{0.6\linewidth}}{\cellcolor[rgb]{0.9,0.9,0.9}{%
	\makecell[{{p{\linewidth}}}]{% 
	  \tt {\tiny [A$\langle$GM]}  
	 guess_feedback: c<red> r<red> a<red> v<red> e<yellow> 
	  } 
	   } 
	   } 
	 & \\ 
 

    \theutterance \stepcounter{utterance}  

    & \multicolumn{4}{p{0.6\linewidth}}{\cellcolor[rgb]{0.8,1,0.9}{%
	\makecell[{{p{\linewidth}}}]{% 
	  \tt {\tiny [A$\rangle$GM]}  
	 guess:deify\\ \tt explanation:The feedback from the previous guess indicates that the letter "e" is in the wrong position. The word "deify" is a verb that means to treat someone as a god or to give them excessive praise. While it may not seem directly related to the concept of hunger, it shares the letter "e" with the word "crave" and has a similar sound and structure to the word "peach." 
	  } 
	   } 
	   } 
	 & & \\ 
 

    \theutterance \stepcounter{utterance}  

    & & & \multicolumn{2}{p{0.3\linewidth}}{\cellcolor[rgb]{0.95,0.95,0.95}{%
	\makecell[{{p{\linewidth}}}]{% 
	  \tt {\tiny [GM$|$GM]}  
	 {'guess': 'deify', 'explanation': 'The feedback from the previous guess indicates that the letter "e" is in the wrong position. The word "deify" is a verb that means to treat someone as a god or to give them excessive praise. While it may not seem directly related to the concept of hunger, it shares the letter "e" with the word "crave" and has a similar sound and structure to the word "peach."'} 
	  } 
	   } 
	   } 
	 & & \\ 
 

    \theutterance \stepcounter{utterance}  

    & & \multicolumn{4}{p{0.6\linewidth}}{\cellcolor[rgb]{0.9,0.9,0.9}{%
	\makecell[{{p{\linewidth}}}]{% 
	  \tt {\tiny [GM$\rangle$B]}  
	 clue:hungry\\ \tt guess:deify\\ \tt explanation:The feedback from the previous guess indicates that the letter "e" is in the wrong position. The word "deify" is a verb that means to treat someone as a god or to give them excessive praise. While it may not seem directly related to the concept of hunger, it shares the letter "e" with the word "crave" and has a similar sound and structure to the word "peach."\\ \tt guess_feedback: c<red> r<red> a<red> v<red> e<yellow> 
	  } 
	   } 
	   } 
	 & \\ 
 

    \theutterance \stepcounter{utterance}  

    & & & \multicolumn{4}{p{0.6\linewidth}}{\cellcolor[rgb]{1,0.85,0.72}{%
	\makecell[{{p{\linewidth}}}]{% 
	  \tt {\tiny [GM$\langle$B]}  
	 agreement:no\\ \tt explanation:the word "deify" does not seem to be related to the concept of hunger, and the guess feedback does not provide any indication that the word "deify" is a possible solution. 
	  } 
	   } 
	   } 
	  \\ 
 

    \theutterance \stepcounter{utterance}  

    & & & \multicolumn{2}{p{0.3\linewidth}}{\cellcolor[rgb]{0.95,0.95,0.95}{%
	\makecell[{{p{\linewidth}}}]{% 
	  \tt {\tiny [GM$|$GM]}  
	 {'agreement': 'no', 'explanation': 'the word "deify" does not seem to be related to the concept of hunger, and the guess feedback does not provide any indication that the word "deify" is a possible solution.'} 
	  } 
	   } 
	   } 
	 & & \\ 
 

    \theutterance \stepcounter{utterance}  

    & & & \multicolumn{2}{p{0.3\linewidth}}{\cellcolor[rgb]{0.95,0.95,0.95}{%
	\makecell[{{p{\linewidth}}}]{% 
	  \tt {\tiny [GM$|$GM]}  
	 Critic disagrees with the Guesser -- Sharing the critic's explanation with the guesser 
	  } 
	   } 
	   } 
	 & & \\ 
 

    \theutterance \stepcounter{utterance}  

    & & \multicolumn{4}{p{0.6\linewidth}}{\cellcolor[rgb]{0.9,0.9,0.9}{%
	\makecell[{{p{\linewidth}}}]{% 
	  \tt {\tiny [A$\langle$GM]}  
	 clue:hungry\\ \tt guess_agreement:no\\ \tt agreement_explanation:the word "deify" does not seem to be related to the concept of hunger, and the guess feedback does not provide any indication that the word "deify" is a possible solution.\\ \tt  
	  } 
	   } 
	   } 
	 & \\ 
 

    \theutterance \stepcounter{utterance}  

    & \multicolumn{4}{p{0.6\linewidth}}{\cellcolor[rgb]{0.8,1,0.9}{%
	\makecell[{{p{\linewidth}}}]{% 
	  \tt {\tiny [A$\rangle$GM]}  
	 guess:grape\\ \tt explanation:The feedback from the previous guess indicates that the letter "e" is in the correct position and the letter "u" is in the wrong position. The word "grape" is a type of fruit that has five letters and shares the letter "e" with the word "crave." It also has a similar sound and structure to the word "peach." 
	  } 
	   } 
	   } 
	 & & \\ 
 

    \theutterance \stepcounter{utterance}  

    & & & \multicolumn{2}{p{0.3\linewidth}}{\cellcolor[rgb]{0.95,0.95,0.95}{%
	\makecell[{{p{\linewidth}}}]{% 
	  \tt {\tiny [GM$|$GM]}  
	 {'guess': 'grape', 'explanation': 'The feedback from the previous guess indicates that the letter "e" is in the correct position and the letter "u" is in the wrong position. The word "grape" is a type of fruit that has five letters and shares the letter "e" with the word "crave." It also has a similar sound and structure to the word "peach."'} 
	  } 
	   } 
	   } 
	 & & \\ 
 

    \theutterance \stepcounter{utterance}  

    & & & \multicolumn{2}{p{0.3\linewidth}}{\cellcolor[rgb]{0.95,0.95,0.95}{%
	\makecell[{{p{\linewidth}}}]{% 
	  \tt {\tiny [GM$|$GM]}  
	 Change in player1's guess\\ \tt guess_before_critic_opinion: deify\\ \tt                                                 critic_agreement: no\\ \tt guess_after_critic_opinion: grape\\ \tt                                                 Proceeding with guess validation 
	  } 
	   } 
	   } 
	 & & \\ 
 

    \theutterance \stepcounter{utterance}  

    & & & \multicolumn{2}{p{0.3\linewidth}}{\cellcolor[rgb]{0.95,0.95,0.95}{%
	\makecell[{{p{\linewidth}}}]{% 
	  \tt {\tiny [GM$|$GM]}  
	 attempts: 6\\ \tt target_word = unfed\\ \tt target_word_clue = hungry\\ \tt guess: grape\\ \tt guess_feedback: g<red> r<red> a<red> p<red> e<yellow> 
	  } 
	   } 
	   } 
	 & & \\ 
 

    \theutterance \stepcounter{utterance}  

    & & & \multicolumn{2}{p{0.3\linewidth}}{\cellcolor[rgb]{0.95,0.95,0.95}{%
	\makecell[{{p{\linewidth}}}]{% 
	  \tt {\tiny [GM$|$GM]}  
	 game_result = LOSS 
	  } 
	   } 
	   } 
	 & & \\ 
 

\end{supertabular}
}

\end{document}
