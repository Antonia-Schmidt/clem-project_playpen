
\documentclass{article}
\usepackage{colortbl}
\usepackage{makecell}
\usepackage{multirow}
\usepackage{supertabular}

\begin{document}

\newcounter{utterance}

\twocolumn

{ \footnotesize  \setcounter{utterance}{1}
\setlength{\tabcolsep}{0pt}
\begin{supertabular}{c@{$\;$}|p{.15\linewidth}@{}p{.15\linewidth}p{.15\linewidth}p{.15\linewidth}p{.15\linewidth}p{.15\linewidth}}

    \# & $\;$A & \multicolumn{4}{c}{Game Master} & $\;\:$B\\
    \hline 

    \theutterance \stepcounter{utterance}  

    & & \multicolumn{4}{p{0.6\linewidth}}{\cellcolor[rgb]{0.9,0.9,0.9}{%
	\makecell[{{p{\linewidth}}}]{% 
	  \tt {\tiny [A$\langle$GM]}  
	 You are given three grids, where each of them is 5 by 5 in size.\\ \tt Grids have empty cells marked with "▢" and filled cells marked with "X".\\ \tt Your task is to generate a referring expression that best describes the target grid while distinguishing it from the two other distractor grids.\\ \tt The first grid is the target grid, and the following two grids are the distractors.\\ \tt \\ \tt Target grid:\\ \tt \\ \tt ▢ X X X ▢\\ \tt ▢ ▢ ▢ X ▢\\ \tt X ▢ X X ▢\\ \tt ▢ X ▢ ▢ ▢\\ \tt ▢ ▢ ▢ X X\\ \tt \\ \tt Distractor grid 1:\\ \tt \\ \tt X X X X ▢\\ \tt ▢ ▢ ▢ ▢ ▢\\ \tt X ▢ X ▢ X\\ \tt ▢ ▢ ▢ ▢ X\\ \tt ▢ ▢ ▢ X X\\ \tt \\ \tt Distractor grid 2:\\ \tt \\ \tt ▢ X X ▢ ▢\\ \tt X ▢ ▢ X ▢\\ \tt X ▢ ▢ ▢ ▢\\ \tt X X ▢ ▢ ▢\\ \tt X X ▢ ▢ X\\ \tt \\ \tt Instruction: Describe the target grid.\\ \tt Generate the referring expression starting with the tag "Expression: " for the given target grid. Omit any other text. 
	  } 
	   } 
	   } 
	 & \\ 
 

    \theutterance \stepcounter{utterance}  

    & \multicolumn{4}{p{0.6\linewidth}}{\cellcolor[rgb]{0.8,1,0.9}{%
	\makecell[{{p{\linewidth}}}]{% 
	  \tt {\tiny [A$\rangle$GM]}  
	 Expression: The grid with a vertical line of three X's in the second column, a horizontal line of three X's in the third row, and a single X in the bottom right corner. 
	  } 
	   } 
	   } 
	 & & \\ 
 

    \theutterance \stepcounter{utterance}  

    & & & \multicolumn{2}{p{0.3\linewidth}}{\cellcolor[rgb]{0.95,0.95,0.95}{%
	\makecell[{{p{\linewidth}}}]{% 
	  \tt {\tiny [GM$|$GM]}  
	 Expression: The grid with a vertical line of three X's in the second column, a horizontal line of three X's in the third row, and a single X in the bottom right corner. 
	  } 
	   } 
	   } 
	 & & \\ 
 

    \theutterance \stepcounter{utterance}  

    & & \multicolumn{4}{p{0.6\linewidth}}{\cellcolor[rgb]{0.9,0.9,0.9}{%
	\makecell[{{p{\linewidth}}}]{% 
	  \tt {\tiny [GM$\rangle$B]}  
	 You are given three grids, where each of them is 5 by 5 in size.\\ \tt Grids have empty cells marked with "▢" and filled cells marked with "X".\\ \tt You are also given a referring expression that describes one of the given grids.\\ \tt Your task is to select the grid that matches the given referring expression.\\ \tt Generate only the number (in text) of the grid that matches the given expression by selecting first, second, or third.\\ \tt \\ \tt First:\\ \tt \\ \tt ▢ X X X ▢\\ \tt ▢ ▢ ▢ X ▢\\ \tt X ▢ X X ▢\\ \tt ▢ X ▢ ▢ ▢\\ \tt ▢ ▢ ▢ X X\\ \tt \\ \tt Second:\\ \tt \\ \tt X X X X ▢\\ \tt ▢ ▢ ▢ ▢ ▢\\ \tt X ▢ X ▢ X\\ \tt ▢ ▢ ▢ ▢ X\\ \tt ▢ ▢ ▢ X X\\ \tt \\ \tt Third:\\ \tt \\ \tt ▢ X X ▢ ▢\\ \tt X ▢ ▢ X ▢\\ \tt X ▢ ▢ ▢ ▢\\ \tt X X ▢ ▢ ▢\\ \tt X X ▢ ▢ X\\ \tt \\ \tt Expression: The grid with a vertical line of three X's in the second column, a horizontal line of three X's in the third row, and a single X in the bottom right corner.\\ \tt Question: Which grid does the expression refer to?\\ \tt Start with the tag "Answer: ", followed by your selection. Omit any other text.\\ \tt  
	  } 
	   } 
	   } 
	 & \\ 
 

    \theutterance \stepcounter{utterance}  

    & & & \multicolumn{4}{p{0.6\linewidth}}{\cellcolor[rgb]{1,0.85,0.72}{%
	\makecell[{{p{\linewidth}}}]{% 
	  \tt {\tiny [GM$\langle$B]}  
	 Answer: Third 
	  } 
	   } 
	   } 
	  \\ 
 

    \theutterance \stepcounter{utterance}  

    & & & \multicolumn{2}{p{0.3\linewidth}}{\cellcolor[rgb]{0.95,0.95,0.95}{%
	\makecell[{{p{\linewidth}}}]{% 
	  \tt {\tiny [GM$|$GM]}  
	 Answer: Third 
	  } 
	   } 
	   } 
	 & & \\ 
 

\end{supertabular}
}

\end{document}
