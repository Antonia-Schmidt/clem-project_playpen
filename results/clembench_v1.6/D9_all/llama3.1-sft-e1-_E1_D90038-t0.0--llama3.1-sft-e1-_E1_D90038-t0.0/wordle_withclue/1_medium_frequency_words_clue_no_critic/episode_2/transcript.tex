
\documentclass{article}
\usepackage{colortbl}
\usepackage{makecell}
\usepackage{multirow}
\usepackage{supertabular}

\begin{document}

\newcounter{utterance}

\twocolumn

{ \footnotesize  \setcounter{utterance}{1}
\setlength{\tabcolsep}{0pt}
\begin{supertabular}{c@{$\;$}|p{.15\linewidth}@{}p{.15\linewidth}p{.15\linewidth}p{.15\linewidth}p{.15\linewidth}p{.15\linewidth}}

    \# & $\;$A & \multicolumn{4}{c}{Game Master} & $\;\:$B\\
    \hline 

    \theutterance \stepcounter{utterance}  

    & & \multicolumn{4}{p{0.6\linewidth}}{\cellcolor[rgb]{0.9,0.9,0.9}{%
	\makecell[{{p{\linewidth}}}]{% 
	  \tt {\tiny [A$\langle$GM]}  
	 You are a language wizard who likes to guess words by using the given rules.\\ \tt \\ \tt Welcome to Wordle! You have six attempts to guess the target word, a valid English word of five lowercase letters (a-z). Please use the tags "guess:" and "explanation:" to provide a concise explanation for each guess.\\ \tt \\ \tt  To help you make an informed guess, you will receive a clue for the word, such as\\ \tt clue: snowy white.\\ \tt \\ \tt Here is an example guess based on the clue:\\ \tt guess: apple\\ \tt explanation: In the fairy tail Snow White, the girl is killed because she eats a poisoned apple. And the word apple has 5 letters.\\ \tt \\ \tt After each guess, your answer will be validated, and you will receive feedback indicating which letters are correct (green), which letters are correct but in the wrong position (yellow), and which letters are incorrect (red). This feedback can be useful in determining which letters to include or exclude in your next guess.\\ \tt \\ \tt For example, the feedback for "apple" might be:\\ \tt guess_feedback: a<yellow> p<yellow> p<green> l<yellow> e<red>\\ \tt \\ \tt The explanation should contain details about how both the guess_feedback and clue work together in tandem to formulate a new guess.\\ \tt \\ \tt Let's begin with your first guess.\\ \tt clue:chin indentation\\ \tt  
	  } 
	   } 
	   } 
	 & \\ 
 

    \theutterance \stepcounter{utterance}  

    & \multicolumn{4}{p{0.6\linewidth}}{\cellcolor[rgb]{0.8,1,0.9}{%
	\makecell[{{p{\linewidth}}}]{% 
	  \tt {\tiny [A$\rangle$GM]}  
	 guess: belly\\ \tt explanation: The clue "chin indentation" suggests that the word is related to the shape or appearance of a chin. A belly button is a small indentation on the navel, but a belly is a part of the body that is not directly related to the chin. However, a belly button is not the most likely guess given the clue. A more plausible guess is a belly button's shape, which is often described as a small indentation. A belly button is a small indentation 
	  } 
	   } 
	   } 
	 & & \\ 
 

    \theutterance \stepcounter{utterance}  

    & & & \multicolumn{2}{p{0.3\linewidth}}{\cellcolor[rgb]{0.95,0.95,0.95}{%
	\makecell[{{p{\linewidth}}}]{% 
	  \tt {\tiny [GM$|$GM]}  
	 {'guess': 'belly', 'explanation': 'The clue "chin indentation" suggests that the word is related to the shape or appearance of a chin. A belly button is a small indentation on the navel, but a belly is a part of the body that is not directly related to the chin. However, a belly button is not the most likely guess given the clue. A more plausible guess is a belly button\'s shape, which is often described as a small indentation. A belly button is a small indentation'} 
	  } 
	   } 
	   } 
	 & & \\ 
 

    \theutterance \stepcounter{utterance}  

    & & & \multicolumn{2}{p{0.3\linewidth}}{\cellcolor[rgb]{0.95,0.95,0.95}{%
	\makecell[{{p{\linewidth}}}]{% 
	  \tt {\tiny [GM$|$GM]}  
	 attempts: 1\\ \tt target_word = cleft\\ \tt target_word_clue = chin indentation\\ \tt guess: belly\\ \tt guess_feedback: b<red> e<yellow> l<yellow> l<red> y<red> 
	  } 
	   } 
	   } 
	 & & \\ 
 

    \theutterance \stepcounter{utterance}  

    & & \multicolumn{4}{p{0.6\linewidth}}{\cellcolor[rgb]{0.9,0.9,0.9}{%
	\makecell[{{p{\linewidth}}}]{% 
	  \tt {\tiny [A$\langle$GM]}  
	 guess_feedback: b<red> e<yellow> l<yellow> l<red> y<red> 
	  } 
	   } 
	   } 
	 & \\ 
 

    \theutterance \stepcounter{utterance}  

    & \multicolumn{4}{p{0.6\linewidth}}{\cellcolor[rgb]{0.8,1,0.9}{%
	\makecell[{{p{\linewidth}}}]{% 
	  \tt {\tiny [A$\rangle$GM]}  
	 guess:house\\ \tt explanation:Given the guess_feedback, we can see that the letters "b" and "y" are incorrect (red), and the letters "e" and "l" are partially correct (yellow). The clue "chin indentation" suggests that the word is related to the shape or appearance of a chin. A house is not typically associated with a chin indentation, but the word "house" has a similar structure to the word "belly". The letters "h" 
	  } 
	   } 
	   } 
	 & & \\ 
 

    \theutterance \stepcounter{utterance}  

    & & & \multicolumn{2}{p{0.3\linewidth}}{\cellcolor[rgb]{0.95,0.95,0.95}{%
	\makecell[{{p{\linewidth}}}]{% 
	  \tt {\tiny [GM$|$GM]}  
	 {'guess': 'house', 'explanation': 'Given the guess_feedback, we can see that the letters "b" and "y" are incorrect (red), and the letters "e" and "l" are partially correct (yellow). The clue "chin indentation" suggests that the word is related to the shape or appearance of a chin. A house is not typically associated with a chin indentation, but the word "house" has a similar structure to the word "belly". The letters "h"'} 
	  } 
	   } 
	   } 
	 & & \\ 
 

    \theutterance \stepcounter{utterance}  

    & & & \multicolumn{2}{p{0.3\linewidth}}{\cellcolor[rgb]{0.95,0.95,0.95}{%
	\makecell[{{p{\linewidth}}}]{% 
	  \tt {\tiny [GM$|$GM]}  
	 attempts: 2\\ \tt target_word = cleft\\ \tt target_word_clue = chin indentation\\ \tt guess: house\\ \tt guess_feedback: h<red> o<red> u<red> s<red> e<yellow> 
	  } 
	   } 
	   } 
	 & & \\ 
 

    \theutterance \stepcounter{utterance}  

    & & \multicolumn{4}{p{0.6\linewidth}}{\cellcolor[rgb]{0.9,0.9,0.9}{%
	\makecell[{{p{\linewidth}}}]{% 
	  \tt {\tiny [A$\langle$GM]}  
	 guess_feedback: h<red> o<red> u<red> s<red> e<yellow> 
	  } 
	   } 
	   } 
	 & \\ 
 

    \theutterance \stepcounter{utterance}  

    & \multicolumn{4}{p{0.6\linewidth}}{\cellcolor[rgb]{0.8,1,0.9}{%
	\makecell[{{p{\linewidth}}}]{% 
	  \tt {\tiny [A$\rangle$GM]}  
	 guess:flute\\ \tt explanation:Given the guess_feedback, we can see that the letters "h", "o", "u", "s" are incorrect (red), and the letter "e" is partially correct (yellow). The clue "chin indentation" suggests that the word is related to the shape or appearance of a chin. A flute is a musical instrument that is not typically associated with a chin indentation. However, the word "flute" has a similar structure to the word 
	  } 
	   } 
	   } 
	 & & \\ 
 

    \theutterance \stepcounter{utterance}  

    & & & \multicolumn{2}{p{0.3\linewidth}}{\cellcolor[rgb]{0.95,0.95,0.95}{%
	\makecell[{{p{\linewidth}}}]{% 
	  \tt {\tiny [GM$|$GM]}  
	 {'guess': 'flute', 'explanation': 'Given the guess_feedback, we can see that the letters "h", "o", "u", "s" are incorrect (red), and the letter "e" is partially correct (yellow). The clue "chin indentation" suggests that the word is related to the shape or appearance of a chin. A flute is a musical instrument that is not typically associated with a chin indentation. However, the word "flute" has a similar structure to the word'} 
	  } 
	   } 
	   } 
	 & & \\ 
 

    \theutterance \stepcounter{utterance}  

    & & & \multicolumn{2}{p{0.3\linewidth}}{\cellcolor[rgb]{0.95,0.95,0.95}{%
	\makecell[{{p{\linewidth}}}]{% 
	  \tt {\tiny [GM$|$GM]}  
	 attempts: 3\\ \tt target_word = cleft\\ \tt target_word_clue = chin indentation\\ \tt guess: flute\\ \tt guess_feedback: f<yellow> l<green> u<red> t<yellow> e<yellow> 
	  } 
	   } 
	   } 
	 & & \\ 
 

    \theutterance \stepcounter{utterance}  

    & & \multicolumn{4}{p{0.6\linewidth}}{\cellcolor[rgb]{0.9,0.9,0.9}{%
	\makecell[{{p{\linewidth}}}]{% 
	  \tt {\tiny [A$\langle$GM]}  
	 guess_feedback: f<yellow> l<green> u<red> t<yellow> e<yellow> 
	  } 
	   } 
	   } 
	 & \\ 
 

    \theutterance \stepcounter{utterance}  

    & \multicolumn{4}{p{0.6\linewidth}}{\cellcolor[rgb]{0.8,1,0.9}{%
	\makecell[{{p{\linewidth}}}]{% 
	  \tt {\tiny [A$\rangle$GM]}  
	 guess:slate\\ \tt explanation:Given the guess_feedback, we can see that the letters "f", "l", "t" are partially correct (yellow), and the letters "u" and "e" are partially correct (yellow). The clue "chin indentation" suggests that the word is related to the shape or appearance of a chin. A slate is a type of material that is not typically associated with a chin indentation. However, the word "slate" has a similar structure 
	  } 
	   } 
	   } 
	 & & \\ 
 

    \theutterance \stepcounter{utterance}  

    & & & \multicolumn{2}{p{0.3\linewidth}}{\cellcolor[rgb]{0.95,0.95,0.95}{%
	\makecell[{{p{\linewidth}}}]{% 
	  \tt {\tiny [GM$|$GM]}  
	 {'guess': 'slate', 'explanation': 'Given the guess_feedback, we can see that the letters "f", "l", "t" are partially correct (yellow), and the letters "u" and "e" are partially correct (yellow). The clue "chin indentation" suggests that the word is related to the shape or appearance of a chin. A slate is a type of material that is not typically associated with a chin indentation. However, the word "slate" has a similar structure'} 
	  } 
	   } 
	   } 
	 & & \\ 
 

    \theutterance \stepcounter{utterance}  

    & & & \multicolumn{2}{p{0.3\linewidth}}{\cellcolor[rgb]{0.95,0.95,0.95}{%
	\makecell[{{p{\linewidth}}}]{% 
	  \tt {\tiny [GM$|$GM]}  
	 attempts: 4\\ \tt target_word = cleft\\ \tt target_word_clue = chin indentation\\ \tt guess: slate\\ \tt guess_feedback: s<red> l<green> a<red> t<yellow> e<yellow> 
	  } 
	   } 
	   } 
	 & & \\ 
 

    \theutterance \stepcounter{utterance}  

    & & \multicolumn{4}{p{0.6\linewidth}}{\cellcolor[rgb]{0.9,0.9,0.9}{%
	\makecell[{{p{\linewidth}}}]{% 
	  \tt {\tiny [A$\langle$GM]}  
	 guess_feedback: s<red> l<green> a<red> t<yellow> e<yellow> 
	  } 
	   } 
	   } 
	 & \\ 
 

    \theutterance \stepcounter{utterance}  

    & \multicolumn{4}{p{0.6\linewidth}}{\cellcolor[rgb]{0.8,1,0.9}{%
	\makecell[{{p{\linewidth}}}]{% 
	  \tt {\tiny [A$\rangle$GM]}  
	 guess:clink\\ \tt explanation:Given the guess_feedback, we can see that the letters "l" and "t" are partially correct (yellow), and the letters "a" and "e" are partially correct (yellow). The clue "chin indentation" suggests that the word is related to the shape or appearance of a chin. A clink is a type of sound that is not typically associated with a chin indentation. However, the word "clink" has a similar structure to 
	  } 
	   } 
	   } 
	 & & \\ 
 

    \theutterance \stepcounter{utterance}  

    & & & \multicolumn{2}{p{0.3\linewidth}}{\cellcolor[rgb]{0.95,0.95,0.95}{%
	\makecell[{{p{\linewidth}}}]{% 
	  \tt {\tiny [GM$|$GM]}  
	 {'guess': 'clink', 'explanation': 'Given the guess_feedback, we can see that the letters "l" and "t" are partially correct (yellow), and the letters "a" and "e" are partially correct (yellow). The clue "chin indentation" suggests that the word is related to the shape or appearance of a chin. A clink is a type of sound that is not typically associated with a chin indentation. However, the word "clink" has a similar structure to'} 
	  } 
	   } 
	   } 
	 & & \\ 
 

    \theutterance \stepcounter{utterance}  

    & & & \multicolumn{2}{p{0.3\linewidth}}{\cellcolor[rgb]{0.95,0.95,0.95}{%
	\makecell[{{p{\linewidth}}}]{% 
	  \tt {\tiny [GM$|$GM]}  
	 attempts: 5\\ \tt target_word = cleft\\ \tt target_word_clue = chin indentation\\ \tt guess: clink\\ \tt guess_feedback: c<green> l<green> i<red> n<red> k<red> 
	  } 
	   } 
	   } 
	 & & \\ 
 

    \theutterance \stepcounter{utterance}  

    & & \multicolumn{4}{p{0.6\linewidth}}{\cellcolor[rgb]{0.9,0.9,0.9}{%
	\makecell[{{p{\linewidth}}}]{% 
	  \tt {\tiny [A$\langle$GM]}  
	 guess_feedback: c<green> l<green> i<red> n<red> k<red> 
	  } 
	   } 
	   } 
	 & \\ 
 

    \theutterance \stepcounter{utterance}  

    & \multicolumn{4}{p{0.6\linewidth}}{\cellcolor[rgb]{0.8,1,0.9}{%
	\makecell[{{p{\linewidth}}}]{% 
	  \tt {\tiny [A$\rangle$GM]}  
	 guess:clues\\ \tt explanation:Given the guess_feedback, we can see that the letters "c" and "l" are correct (green), and the letters "i", "n", "k" are incorrect (red). The clue "chin indentation" suggests that the word is related to the shape or appearance of a chin. A clue is a hint or a piece of information that is not typically associated with a chin indentation. However, the word "clues" has a similar 
	  } 
	   } 
	   } 
	 & & \\ 
 

    \theutterance \stepcounter{utterance}  

    & & & \multicolumn{2}{p{0.3\linewidth}}{\cellcolor[rgb]{0.95,0.95,0.95}{%
	\makecell[{{p{\linewidth}}}]{% 
	  \tt {\tiny [GM$|$GM]}  
	 {'guess': 'clues', 'explanation': 'Given the guess_feedback, we can see that the letters "c" and "l" are correct (green), and the letters "i", "n", "k" are incorrect (red). The clue "chin indentation" suggests that the word is related to the shape or appearance of a chin. A clue is a hint or a piece of information that is not typically associated with a chin indentation. However, the word "clues" has a similar'} 
	  } 
	   } 
	   } 
	 & & \\ 
 

    \theutterance \stepcounter{utterance}  

    & & & \multicolumn{2}{p{0.3\linewidth}}{\cellcolor[rgb]{0.95,0.95,0.95}{%
	\makecell[{{p{\linewidth}}}]{% 
	  \tt {\tiny [GM$|$GM]}  
	 attempts: 6\\ \tt target_word = cleft\\ \tt target_word_clue = chin indentation\\ \tt guess: clues\\ \tt guess_feedback: c<green> l<green> u<red> e<yellow> s<red> 
	  } 
	   } 
	   } 
	 & & \\ 
 

    \theutterance \stepcounter{utterance}  

    & & & \multicolumn{2}{p{0.3\linewidth}}{\cellcolor[rgb]{0.95,0.95,0.95}{%
	\makecell[{{p{\linewidth}}}]{% 
	  \tt {\tiny [GM$|$GM]}  
	 game_result = LOSS 
	  } 
	   } 
	   } 
	 & & \\ 
 

\end{supertabular}
}

\end{document}
